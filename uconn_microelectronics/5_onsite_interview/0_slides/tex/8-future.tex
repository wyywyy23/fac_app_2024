%!TEX root = ../research_overview.tex

\section{Future Directions}

\subsection{Photonics for RF}

\begin{frame}{Photonics-Enabled Massive Antenna Arrays}

    \only<2-2|handout:0>{\centering\includegraphics[width=0.8\textwidth]{fig/rf-1_compressed.pdf}}%
    \only<3-3>{\centering\includegraphics[width=0.8\textwidth]{fig/rf-2_compressed.pdf}}%
    \note<1-1>[item]{So, looking into the future, one exciting direction I would like to pursue is to find more applications of integrated photonics technology at the intersection of communication and computation.}
    \note<2-2>[item]{For example, for next-G wireless networks, massive distributed antenna arrays will be a key technology to provide both the high bandwidth and high angular resolution at the same time. And it requires ultra high-throughput signal processing and low-latency communication at the antenna backend.}
    \note<2-2>[item]{So, we can imagine having an RF antenna element directly interfaced with a photonic chip that can perform high-performance data processing locally,}
    \note<3-3>[item]{but can also communication with other elements through high-bandwidth optical interconnects and switches.}
    \note<3-3>[item]{This would open up the possibility of having, perhaps, a million-element antenna array in the future.}

\end{frame}

\subsection{3D Photonics}

\begin{frame}{3D Photonics for Transformative System Connectivity}

    \only<2-2|handout:0>{\centering\includegraphics[width=\textwidth]{fig/3d-1_compressed.pdf}}%
    \only<3-3>{\centering\includegraphics[width=\textwidth]{fig/3d-2_compressed.pdf}}%
    \note<1-1>[item]{Another exciting direction that I want to pursue is to explore the possibility of 3D optical routing and interconnections.}
    \note<2-2>[item]{This is really to break away from today's limitations of planar optical routing where every photonic component or waveguide needs to be on the same plane. And the bandwidth density you can have out of a chip is fundamentally limited by the fiber pitch you can have along a two dimensional shoreline.}
    \note<3-3>[item]{So, if we can route optical signals in the third dimension, and have dense optical I/Os that are accessible from essentially all directions, we would be able to transform the paradigm of optically connected systems, which currently relies on optical fibers, to something that uses an interposer-like photonic substrate.}
    \note<3-3>[item]{And this would potentially enable wafer-scale or panel-scale computation where each wafer or panel-sized system is capable of what a supercomputer can do today.}
    % \note<3-3>[item]{And I look forward to, and think that I would highly benefit from, collaborating with the faculty here who are experts in circuit design, such as Professor Kim and Professor Harjani, to explore electronic-photonic co-design and integration; and with top system architects, such as Professor Cao}


\end{frame}

\section{Acknowledgements}

\begin{frame}{Collaborators and Funding}
    \centering\includegraphics[width=\textwidth]{fig/ack_compressed.pdf}
    \large yw3831@columbia.edu
    \note[item]{And with that, I would like to thank you all for your attention, and I'm happy to take any questions.}
\end{frame}