%!TEX root = ../research_overview.tex

\section{Energy Efficiency}

\subsection{Link Budget}

\begin{frame}{Link Budget Analysis}

    \only<2-2>{\centering\includegraphics[width=0.85\textwidth]{fig/budget_compressed.pdf}}%
    \note<1-1>[item]{Now, let me have a few words on the energy efficiency of the link.}
    \note<2-2>[item]{And this energy estimation really starts from doing a link budget analysis, where I calculate the minimum power required from the comb laser to achieve a target bit error rate, which is 10 to the minus 12. This optical power from the comb will need to traverse a series of optical losses and eventually satisfy the receiver sensitivity.}
    \note<2-2>[item]{And I'd point that all the optical losses listed here are characterized from measurement data.}

\end{frame}

\subsection{Breakdown}

\begin{frame}{Energy Breakdown}

    \only<1-1>{\centering\includegraphics[width=\textwidth]{fig/energy_compressed.pdf}}%
    \note<1-1>[item]{And then, the laser power consumption is calculated assuming a 15\% wall-plug efficiency, and the total energy consumption here include not only the laser, but also the driver, receiver, and the thermal control. And it takes about 300 fJ/b with thermal undercut, which is foundry process that selectively removes part of the substrate around and below a device to improve the thermal tuning efficiency. And I'd like to emphasize that this 300 fJ/b is able to carry that data to almost any distance in the system.}

\end{frame}


\begin{frame}{Return to Metrics}

    \only<1-1|handout:0>{\centering\includegraphics[width=0.8\textwidth]{fig/fom-4_compressed.pdf}}%
    \only<2-2>{\centering\includegraphics[width=0.8\textwidth]{fig/fom-future_compressed.pdf}}%
    \note<1-1>[item]{Now, let me return to this link metric and look at what we have achieved so far, and ask if we can further improve the bandwidth density and energy efficiency.}
    \note<2-2>[item]{And I'd say yes, at least another magnitude, and this is just the beginning, because we have lots of room to improve at both device and integration levels, like for example,}

\end{frame}

\begin{frame}{Paths to Improvement}

    \only<1-1|handout:0>{\centering\includegraphics[width=0.8\textwidth]{fig/path-improve-1_compressed.pdf}}%
    \only<2-2|handout:0>{\centering\includegraphics[width=0.8\textwidth]{fig/path-improve-2_compressed.pdf}}%
    \only<3-3>{\centering\includegraphics[width=0.9\textwidth]{fig/path-improve-3_compressed.pdf}}%
    \note<1-1>[item]{we have disk designs that can be driven faster at an even lower Vpp.}
    \note<2-2>[item]{our link architecture is scalable and we're looking at 128 channels per link.}
    \note<3-3>[item]{and with foundries starting to offer thick silicon nitride layers, we can potentially have an integrated comb source that lowers the optical losses and the energy consumption.}
    \note<3-3>[item]{With that, I would like to switch gears now and talk about how}

\end{frame}
