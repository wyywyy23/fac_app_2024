%!TEX root = ../research_overview.tex

\section{Approach}

\subsection{Embedded Photonics}

\begin{frame}{Bringing Photonics into Computing Sockets}

    \note<1-1>[item]{So, here is my approach, which is to bring photonic technologies into the computing sockets.}
    \note<2-2>[item]{If we look at today's pluggable optical transceivers, we see this long electrical wire, which can be tens of centimeters long, that is still needed between the computing chip and the optical interface. To drive this electrical wiring at something like 100 Gb/s is where the energy efficiency really starts to degrade.}
    \note<3-3>[item]{So, my research has focused on integrating the photonics data input/output directly into the computing socket, because once the data is in the optical domain, the bandwidth and energy become virtually independent of distance, which means it can travel from centimeters to hundreds of meters with roughly the same bandwidth and energy efficiency. So the key is to really have this electrical to optical conversion as close as possible to where the data is generated.}
    \note<3-3>[item]{But this is a very challenging task, because we not only need to figure out the integration and packaging strategies, which I'll talk about later, but we also need to find the optimal photonic link architecture that seamlessly interface to the electronics.}

\end{frame}

\subsection{Channel Parallelism}

\begin{frame}{Massive Wavelength Parallelism}

    \note<1-1>[item]{One example of this photonic-electronic co-design is to figure out how many parallel wavelength channels we need to have per link in order to achieve the target bandwidth with the best energy efficiency.}
    \note<2-2>[item]{So, what I'm showing here is a combined metric of bandwidth density and energy efficiency, where the x-axis is the number of parallel wavelength channels per 1 Tb/s bandwidth. In other words, to the left of the x-axis, there are fewer wavelength channels, each operating at a higher data rate, and to the right, there are more wavelength channels, each operating at a lower data rate.}

\end{frame}