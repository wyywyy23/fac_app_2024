%!TEX root = ../research_overview.tex

\section{Integration and Packaging}

\subsection{Approach}

\begin{frame}{Hybrid 2.5D/3D Integration}

    \only<2-2|handout:0>{\centering\includegraphics[width=0.8\textwidth]{fig/integration-1_compressed.pdf}}%
    \only<3-3>{\centering\includegraphics[width=0.8\textwidth]{fig/integration-2_compressed.pdf}}%
    \note<1-1>[item]{So, that was the link design. To get the bandwidth density though, we need to think about how to integrate the photonics with electronic drivers and package with the compute chip.}
    \note<2-2>[item]{What I'm showing here are some of the integration and packaging options. And together with Cornell University and Intel,}
    \note<3-3>[item]{we made the first demonstration of a co-packaged optical I/O assembly that takes the bottom-right approach, which is a hybrid 2.5D/3D integration.}
    \note<3-3>[item]{More specifically, we developed a multi-chip package, or MCP, that has the photonic chiplet and electronic chiplet stacked on top of each other, and then this sub-assembly is further integrated with the compute chip through a silicon interposer.}
    \note<3-3>[item]{Compared to other approaches that use a monolithic chiplet for both photonic components and their electronic drivers, having two separate chiplets for photonics and electronics allows us to optimize both using different process nodes, and also achieve a much higher density by using flip-chip bonding with a smaller microbump pitch.}
    \note<3-3>[item]{And the 2.5D integration with the compute chip requires no changes to the compute chip itself, which is a big advantage for system prototyping.}

\end{frame}

\subsection{Multi-Chip Package}

\begin{frame}{96\,Tb/s Multi-Chip Package}

    \only<1-1|handout:0>{\centering\includegraphics[width=\textwidth]{fig/mcp-1_compressed.pdf}}%
    \only<2-2>{\centering\includegraphics[width=\textwidth]{fig/mcp-2_compressed.pdf}}%
    \vspace{-1em}%
    \begin{center}%
        \onslide<1->{\fullcite{wangCoDesignedSilicon2024}}%
    \end{center}%
    \note<1-1>[item]{So, here's what the multi-chip package looks like. And I actually have one here, so you can get an idea of how large it is.}
    \note<2-2>[item]{Inside the package, there is an Intel FPGA that acts as the compute chip, and three of these photonic-electronic sub-assemblies co-integrated through Intel's EMIB technology.}
    \note<2-2>[item]{And this MCP is designed to provide a 96 Tb/s bidirectional bandwidth into and out of the package, and I'll show you how this number is achieved.}

\end{frame}

\begin{frame}{4\,Tb/s/mm Shoreline Bandwidth Density}

    \note<1-1>[item]{Here is what it looks like under the hood. In this collaborative work, I designed the three photonic chips, each bonded to an electronic driver chip designed at Cornell University, and co-packaged with the Intel FPGA sitting right next to it.}
    \note<1-1>[item]{We can see how the three photonic-electronic sub-assemblies match the shoreline of the FPGA chip.}
    \note<1-1>[item]{Among the 48 fibers that are attached to each photonic chiplet, 16 of them are comb input, and 32 of them are data I/Os with 1 Tb/s per fiber bandwidth capacity. Dividing this by the 8 mm shoreline of each chiplet, we get a shoreline bandwidth density of 4 Tb/s/mm.}

\end{frame}

\begin{frame}{Photonic Chiplet Design}

    \note<1-1>[item]{And here is the detailed floorplan of the photonic chiplet that I designed.}
    \note<1-1>[item]{Each photonic chiplet has a footprint of roughly 8 mm by 8 mm, and it contains a total of 16 transmitter links and 16 receiver links. Each link has the 4 by 16 architecture that I showed earlier, and delivers 1 Tb/s bandwidth.}
    \note<2-2>[item]{In total, this photonic chiplet integrates over 2,000 microresonator modulators and filters with a 55 micron bump pitch. This dense layout is what really enables the high bandwidth density, and is only possible with the use of 3D integration of electronics and photonics.}

\end{frame}
