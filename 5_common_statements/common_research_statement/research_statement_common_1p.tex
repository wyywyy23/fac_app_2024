Data-intensive applications are challenging computational hardware limits as transistors stop scaling, emphasizing the need for \emph{parallel architectures} and \emph{heterogeneous integration}~\cite{iyerHeterogeneousIntegrationPerformance2016}, as well as efficient data movement in hyperscale systems. Optical interconnects, particularly in silicon photonics (SiPh), provide an energy-efficient, cost-effective, and scalable solution leveraging decades of microelectronics manufacturing expertise~\cite{millerRationaleChallengesOptical2000,sorefPresentFutureSilicon2006}. The recent large-scale integration of SiPh components on a single chip~\cite{shekharRoadmappingNextGeneration2024} has opened up opportunities at the intersection of photonics, electronics, and computing systems. During my Ph.D. and postdoc, I contributed significantly to SiPh in areas including electronic-photonic design automation (EPDA)~\cite{wuCompactModelingCircuitlevel2017,zhangCompactModelingSilicon2017,jamesFlexibleProcessAwareCompact2022,jamesProcessVariationAwareCompact2023a}, variation characterization and mitigation~\cite{wuPairingMicroringbasedSilicon2018,wangEnergyefficientChannelAlignment2018,wangTamingEmergingDevices2019,wangBidirectionalTuningMicroringbased2019,wangCharacterizationApplicationsSpatial2020,wangEnergyEfficiencyYield2021}, broadband filters and dispersion engineering~\cite{wangDispersionEngineeredFabricationRobustSOI2023,wangIntegratedCompactTunable2023,parsonsOFC25}, ultra-scalable link architectures~\cite{wangScalableArchitectureSubpJ2023,novickHighbandwidthDensitySilicon2023}, automated link control~\cite{wangAutomatedTuningRingAssisted2024,wangInterleaverTuning,wangOFC25}, channel-dependent optimization~\cite{novickIntegratedPhotonicResonant2024,gopalEqualization2024}, network-application co-optimization~\cite{wangTaskMappingAssistedLaser2019,wangTrafficAdaptivePowerReconfiguration2021,michelogiannakisEfficientIntraRackResource2023,wuWavelengthReconfigurableTransceiver2024,wuFlexibleSiliconPhotonic2024}, and photonics-enabled data processing~\cite{naumanOFC25,zypmanDSP}. Most notably, my postdoctoral research at \mySchoolShort{} achieved groundbreaking integration of photonic input/output (I/O) chiplets with electronic drivers and computing resource, leveraging Kerr comb\textendash{}driven dense wavelength-division multiplexing (DWDM)~\cite{wangSiliconPhotonicsChip2024,wangCoDesignedSiliconPhotonics2024,RovinskiISCAS25}. This prototype demonstrated record shoreline bandwidth density and energy efficiency, advancing the interconnect figure of merit (FOM) by two orders of magnitude (Fig.~\ref{fig:embedded_photonics}a). This work has led to subsequent funding from both government and industry sources, as well as extensive collaborations with industry and academia, enabling prototype refinement and exploration of transformative connectivity paradigms.

\begin{wrapfigure}[32]{r}{0.35\textwidth}
    \vspace{-3em}
    \begin{center}
        \includegraphics[width=\linewidth]{../../6_figures/rs_fig_1_empho_3dpho_comppho_compressed.pdf}
    \end{center}
    \vspace{-1em}
    \caption{(a)~My contribution to interconnect FOM. (b)~Envisioned 3D optical connectivity. (c)~Photonics-enabled computing within data movement.}
    \label{fig:embedded_photonics}
\end{wrapfigure}

\textbf{Transforming Chip-to-Chip Connectivity.} Today's computing clusters face connectivity bottlenecks due to bandwidth limitations of electrical wires over distance. As part of the DARPA PIPES program, I led the photonics design of an optical I/O chiplet with massive wavelength parallelism, achieving a shoreline bandwidth density beyond 4\,Tbps/mm at sub-pJ/bit energy, successfully advancing the program to Phase 3 and substantially contributed to the winning of an SRC JUMP 2.0 and a Samsung GRO grant. In these subsequent projects, I have identified a clear pathway to another order of magnitude improvement in the link FOM through on-chip laser integration and wafer-scale substrate undercut, marking a significant leap toward the goal in Fig.~\ref{fig:embedded_photonics}a. With ultra-low modulator driving voltages (<\,0.2\,V) demonstrated, an exciting future extension comes from 3D-integrated computing resources directly driving the embedded photonics. This envisioned architecture aligns optical channel data rates with processor/memory speeds, allowing for efficient pooling of spatially-distanced resources and eliminating data locality concerns in large-scale parallel computing workloads.

Looking further ahead, I aim to explore a new paradigm of chip-to-chip connectivity that lifts the restrictions of today's planar optical routing, envisioning a dense 3D optical system architecture (Fig.~\ref{fig:embedded_photonics}b) that enables any-to-any optical interconnections between active photonic chips. The chips can be integrated via metal-metal hybrid bonding, permitting high-density electrical connections while
additionally providing high alignment accuracy for minimal optical misalignment. This vision, shared at the 2023 DARPA ERI Summit, opens new collaboration opportunities across device, system, packaging, and design automation fields.

\textbf{Computing within Data Movement.} Emerging SiPh applications blur communication and computation boundaries by leveraging optical signals for operations like matrix-vector multiplications~\cite{taitMicroringWeightBanks2016,shenDeepLearningCoherent2017}. However, current approaches face challenges with scalability, precision, and control complexity, in addition to the need for digital-to-analog and analog-to-digital conversions at the input/output of these architectures. We have proposed and validated a novel architecture that directly interfaces with the logic die of the memory chip, enabling multiply-accumulate (MAC) operations as data is in motion (Fig.~\ref{fig:embedded_photonics}c). This architecture encodes each bit place of the MAC result as an analog light intensity and counts the number of 1's through a photonic analog-to-digital converter (ADC). As each microresonator operates in a binary mode, this architecture is inherently more robust to fabrication and environmental variations than those leveraging analog intensity/phase modulation. Along this line, I am actively exploring the potential of integrating photonic data processing with not only DWDM, but also other architectures such as mode division multiplexing (MDM), as well as applications beyond AI/machine learning, such as using photonic data links as the back plane of large-scale antenna arrays and perform signal processing within data movement.

\textbf{Ecosystem, Funding, and Collaborations.} My extensive experience with the fabless SiPh ecosystem includes nine tape-outs and leading a dedicated 300\,mm AIM Photonics run. I have worked on DARPA and ARPA-E projects, both advancing to Phase 3, and have contributed to over ten grant proposals including recent CHIPS ACT calls on packaging and digital twins. Direct engagement with program managers has provided me with insights into successful proposals and program advancement criteria. My close relationships across academia and industry will provide immediate research avenues that expand on previous work and enable entirely new directions. At \appSchoolDeptShort{}, I additionally envision the opportunity for new collaborations with the current faculty, including:
\begin{enumerate*}[label=(\roman*)]
    \appCollab{}
\end{enumerate*}