The exponential growth of data-intensive applications is pushing the limits of computational hardware. With Moore's Law drawing to a close as silicon transistors reach their physical limits, monolithic chip solutions are fundamentally limited by their ability to scale. Rather than relying on continued miniaturization, the future of computing lies in \emph{parallel architectures} and \emph{heterogeneous integration} to meet the growing demands for performance, yield, and diversified functionality~\cite{iyerHeterogeneousIntegrationPerformance2016}. In modern hyperscale computing infrastructures, such as data centers and high-performance computing systems, the interconnects among heterogeneous computing resources have become integral to the overall system performance, as efficient data movement is critical to avoid data-starved processing units (PUs). Notably, with the rise of large machine learning models comprising billions to trillions of parameters, PUs spatially separated by up to a kilometer need to communicate\textemdash{}a distance scale at which electrical signaling over metal wires consumes excessive energy due to distance-dependent losses. A prevailing solution to this challenge is to leverage the properties of light to send data with distance-agnostic losses and high parallelism through optical interconnects~\cite{millerRationaleChallengesOptical2000}. While a significant amount of research exists in using exotic material platforms for realizing specialized optoelectronic devices, the most elegant solution is to leverage the mature, scalable, and cost-effective precision tooling, developed over decades for the electronics industry, to implement optical transceivers in silicon\textemdash{}the motivation that sparked the field of \emph{silicon photonics} (SiPh)~\cite{sorefPresentFutureSilicon2006}.

After modest initial growth, SiPh has finally seen the promise of large-scale integration in the past decade with thousands to millions of optical components on a single chip~\cite{shekharRoadmappingNextGeneration2024}, following the maturation of 300\,mm SiPh-wafer foundry offerings. The move from individual devices to complex circuits opens up a wealth of interdisciplinary research opportunities at the intersection of photonics, electronics, and computing systems. During the nine years of my \myDegree{} and postdoctoral research, I have actively contributed to the literature in diverse yet synergistic areas of SiPh, including electronic-photonic design automation~\cite{wuCompactModelingCircuitlevel2017,zhangCompactModelingSilicon2017,jamesFlexibleProcessAwareCompact2022,jamesProcessVariationAwareCompact2023a}, variation characterization and mitigation~\cite{wuPairingMicroringbasedSilicon2018,wangEnergyefficientChannelAlignment2018,wangTamingEmergingDevices2019,wangBidirectionalTuningMicroringbased2019,wangCharacterizationApplicationsSpatial2020,wangEnergyEfficiencyYield2021}, broadband filters and dispersion engineering~\cite{wangDispersionEngineeredFabricationRobustSOI2023,wangIntegratedCompactTunable2023,parsonsOFC25}, ultra-scalable link architectures~\cite{wangScalableArchitectureSubpJ2023,novickHighbandwidthDensitySilicon2023}, automated link control~\cite{wangAutomatedTuningRingAssisted2024,wangInterleaverTuning,wangOFC25}, channel-dependent optimization~\cite{novickIntegratedPhotonicResonant2024,gopalEqualization2024}, 2.5D/3D integration~\cite{wangSiliconPhotonicsChip2024,wangCoDesignedSiliconPhotonics2024,RovinskiISCAS25}, network-application co-optimization~\cite{wangTaskMappingAssistedLaser2019,wangTrafficAdaptivePowerReconfiguration2021,michelogiannakisEfficientIntraRackResource2023,wuWavelengthReconfigurableTransceiver2024,wuFlexibleSiliconPhotonic2024}, and photonics-enabled computing and signal processing~\cite{naumanOFC25,zypmanDSP}.


\begin{wrapfigure}{r}{0.4\textwidth}
    \begin{center}
        \includegraphics[width=\linewidth]{../../6_figures/rs_fig_1_empho_compressed.pdf}
    \end{center}
    \caption{Hello, world!}
\end{wrapfigure}

Most notably,

\kant[1-6]