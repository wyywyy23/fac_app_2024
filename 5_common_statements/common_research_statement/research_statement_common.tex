The exponential growth of data-intensive applications is pushing the limits of computational hardware. With Moore's Law drawing to a close as silicon transistors reach their physical limits, monolithic chip solutions are fundamentally limited by their ability to scale. Rather than relying on continued miniaturization, the future of computing lies in \emph{parallel architectures} and \emph{heterogeneous integration} to meet the growing demands for performance, yield, and diversified functionality~\cite{iyerHeterogeneousIntegrationPerformance2016}. In modern hyperscale computing infrastructures, such as data centers and high-performance computing systems, the interconnects among heterogeneous computing resources have become integral to the overall system performance, as efficient data movement is critical to avoid data-starved processing units (PUs). Notably, with the rise of large machine learning models comprising billions to trillions of parameters, PUs spatially separated by up to kilometers need to communicate\textemdash{}a distance scale at which electrical signaling over metal wires consumes excessive energy due to distance-dependent losses. A prevailing solution to this challenge is to leverage the properties of light to send data with distance-agnostic losses and high parallelism through optical interconnects~\cite{millerRationaleChallengesOptical2000}. While a significant amount of research exists in using exotic material platforms for realizing specialized optoelectronic devices, the most elegant solution is to leverage the mature, scalable, and cost-effective precision tooling, developed over decades for the electronics industry, to implement optical transceivers in silicon\textemdash{}the motivation that sparked the field of \emph{silicon photonics} (SiPh)~\cite{sorefPresentFutureSilicon2006}.

\begin{wrapfigure}{r}{0.4\textwidth}
    \vspace{-1em}
    \begin{center}
        \includegraphics[width=\linewidth]{../../6_figures/rs_fig_1_empho_compressed.pdf}
    \end{center}
    \caption{(a)~FOM comparison between pluggable, existing CPO, and my research, showing key enablers of FOM leaps toward future goal. Data sources given in~\cite{wangCoDesignedSiliconPhotonics2024}. (b)~Illustrative integration stack with pluggable optics. (c)~Illustrative integration stack with CPO and our 96\,Tbps MCP prototype. (d)~Envisioned computing node with embedded photonics, where optical modulators are directly driven by electrical I/Os of processor or memory chips.}
    \label{fig:embedded_photonics}
    \vspace{-2.5em}
\end{wrapfigure}

After modest initial growth, SiPh has finally seen in the past decade the promise of large-scale integration with thousands to millions of optical components on a single chip~\cite{shekharRoadmappingNextGeneration2024}, following the maturation of 300\,mm SiPh-wafer foundry offerings. The move from individual devices to complex circuits opens up a wealth of interdisciplinary research opportunities at the intersection of photonics, electronics, and computing systems. During the nine years of my \myDegree{} and postdoctoral research, I have actively contributed to the literature in diverse yet synergistic areas of SiPh, including electronic-photonic design automation (EPDA)~\cite{wuCompactModelingCircuitlevel2017,zhangCompactModelingSilicon2017,jamesFlexibleProcessAwareCompact2022,jamesProcessVariationAwareCompact2023a}, variation characterization and mitigation~\cite{wuPairingMicroringbasedSilicon2018,wangEnergyefficientChannelAlignment2018,wangTamingEmergingDevices2019,wangBidirectionalTuningMicroringbased2019,wangCharacterizationApplicationsSpatial2020,wangEnergyEfficiencyYield2021}, broadband filters and dispersion engineering~\cite{wangDispersionEngineeredFabricationRobustSOI2023,wangIntegratedCompactTunable2023,parsonsOFC25}, ultra-scalable link architectures~\cite{wangScalableArchitectureSubpJ2023,novickHighbandwidthDensitySilicon2023}, automated link control~\cite{wangAutomatedTuningRingAssisted2024,wangInterleaverTuning,wangOFC25}, channel-dependent optimization~\cite{novickIntegratedPhotonicResonant2024,gopalEqualization2024}, network-application co-optimization~\cite{wangTaskMappingAssistedLaser2019,wangTrafficAdaptivePowerReconfiguration2021,michelogiannakisEfficientIntraRackResource2023,wuWavelengthReconfigurableTransceiver2024,wuFlexibleSiliconPhotonic2024}, and photonics-enabled computing and signal processing~\cite{naumanOFC25,zypmanDSP}.

Most notably, my postdoctoral research at \mySchoolShort{} made essential contributions to the first demonstration of a hybrid 2.5D/3D integration of photonic input/output (I/O) chiplets with
flip-chip bonded electronic drivers and a state-of-the-art field-programmable gate array (FPGA)~\cite{wangSiliconPhotonicsChip2024,wangCoDesignedSiliconPhotonics2024,RovinskiISCAS25}. The prototype demonstrated unprecedented shoreline bandwidth density and energy efficiency leveraging Kerr comb\textendash{}driven dense wavelength-division multiplexing (DWDM) and advanced packaging to send/receive massively parallel data streams right at the edge of the processor chip, advancing the link figure of merit (FOM) by two orders of magnitude compared to existing co-packaged optics (CPO) solutions (Fig.~\ref{fig:embedded_photonics}a). These significant results have also substantially contributed to the winning of several subsequent research grants from both government and industry, and led to extensive collaborations with leading industrial partners and academic institutions to refine system prototypes. An overview of this work is presented in the next section alongside future research directions for redefining chip-to-chip connectivity. The subsequent section envisages applications of photonics where computing and signal processing functionalities are integrated within data movement, seeking a paradigm shift of computing systems. Finally, I provide an overview of ecosystem, collaborations, and funding required to realize these visions and a pathway to fulfilling these requirements leveraging my expertise, experience, and network.

\section*{Transforming Chip-to-Chip Connectivity}

A major connectivity bottleneck faced by today's computing clusters is the link bandwidth discrepancy across the network hierarchy. As data travels longer distances, the available bandwidth drops by up to two orders of magnitude to avoid the prohibitive energy required for long-distance electrical signaling. Conventional pluggable optical transceivers (Fig.~\ref{fig:embedded_photonics}b) only extend the reach of the signals, but are fundamentally limited by the excessive power and area required for EO/OE conversion, primarily due to the long electrical traces on-board and the need for high data rates per lane.

As part of the DARPA PIPES program, I spearheaded the photonics design for a photonic I/O chiplet that aims to substantially improve the link FOM, defined as the bandwidth density\textendash{}energy efficiency product [Gbps/mm \texttimes{} (pJ/b)$^{-1}$] (Fig.~\ref{fig:embedded_photonics}a). The link architecture uniquely leverages chip-scale microresonator-based Kerr frequency combs, which are capable of generating hundreds of evenly-spaced low-noise wavelength channels from a single continuous-wave (CW) laser source with high efficiency. This allows achieving a high aggregate data rate with massive wavelength parallelism at a modest data rate per channel\textemdash{}a key enabler of high energy efficiency. Each photonic chiplet, measuring 8.1\,mm \texttimes{} 8.6\,mm, densely integrates over \num{2000} microresonator modulators and filters to deliver a bidirectional bandwidth greater than 32\,Tbps with a 4\,Tbps/mm shoreline bandwidth density. Multi-chip package (MCP) prototypes (Fig.~\ref{fig:embedded_photonics}c) were also developed in collaboration with Intel and Cornell University. Each MCP comprises three electronic driver chiplets fabricated in Intel's 22\,nm process, flip-chip bonded to three photonic transceiver chiplets, and further integrated with an Intel Stratix 10 FPGA using Intel's Embedded Multi-Die Interconnect Bridge (EMIB) technology. This MCP prototype is the first to demonstrate a hybrid 2.5D/3D\textendash{}integrated CPO solution promising of a 96\,Tbps bidirectional bandwidth at a sub-pJ/b energy efficiency. As Fig.~\ref{fig:embedded_photonics}a shows, these results represent a two-order-of-magnitude improvement in the link FOM compared to state-of-the-art CPO solutions, and a \num{1000}\texttimes{} improvement over the best-in-class pluggable transceivers. These results have led the successful transition of the DARPA PIPES program into Phase 3, and substantially contributed to the winning of an SRC JUMP 2.0 grant and a Samsung GRO grant. In these subsequent projects, I have worked closely with leading industrial companies (Intel, Samsung, and Applied Materials), academic institutions (Cornell, MIT, and UIUC), and foundry partners (AIM Photonics and GlobalFoundries) to push the limits of interconnect bandwidth density and energy efficiency through electronic-photonic co-design and heterogeneous integration. A preliminary link budget analysis based on measured devices have shown great promise for another order of magnitude improvement in the link FOM achievable through on-chip laser integration and wafer-scale substrate undercut, marking a significant leap toward the goal in Fig.~\ref{fig:embedded_photonics}a.