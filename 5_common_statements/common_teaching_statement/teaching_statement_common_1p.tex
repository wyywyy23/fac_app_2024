My teaching philosophy is rooted in the belief of the transformative power of education, profoundly shaped by inspiring mentors of mine. This belief is reinforced by today's abundance of open online courses from top universities. This democratization of higher education has emphasized the importance of an instructor's role in guiding students through the learning process, rather than simply delivering information. As cliché as it may sound, what still remains true as a teaching philosophy is to teach students \emph{how} to think, rather than \emph{what} to think. With the increasingly diversified student body and teaching modality, it also requires a more nuanced approach to balance the act of engaging, conveying, and inspiring. My pedagogical practices are guided by introspective questions ensuring that lectures are engaging, slides are self-contained and comprehensible, and takeaways remain consistent across contexts. Having worked with device physicists and system engineers, I integrate a “pragmatic first principles” approach to problem-solving, bridging analytical rigor with practical constraints. This philosophy aligns seamlessly with \appSchoolDeptShort{}'s interdisciplinary curriculum, preparing students to tackle real-world challenges.

I additionally value the role of good-quality visualization for improving teaching effectiveness, leveraging human brain's rapid image processing ability. This capability enables students to extract information from visual aids alongside verbal explanations far more efficiently than text alone, allowing for profound engagement in class. In the realm of STEM education, where abstract theories and complex equations can be overwhelming, visualization serves as a vital bridge, translating intricate ideas into comprehensible and memorable images. It also elegantly addresses the pedagogical challenge of conveying essential concepts without resorting to simply reading from the slides\textemdash a practice that hinders critical thinking. Beyond the immediate classroom benefits, the ability to visualize data and concepts is an indispensable skill for students, one that is increasingly critical in both their academic pursuits and future research careers.

\section*{Teaching Experience}
As a teaching assistant for “ECE 153B: Sensor \& Peripheral Interface Design” at \myPhDSchoolShort{}, I led lab sessions and office hours, guiding students through debugging challenges in hardware and software design. I worked closely with students at varying skill levels, ensuring successful project demonstrations through individualized mentorship. Witnessing students' final projects\textemdash{}many of which stemmed from collaborative brainstorming and iterative refinement\textemdash{}was one of the most rewarding aspects of my teaching experience.

I have shared my experiences and lessons on preparing accessible visual content with graduate students at \mySchoolShort{}. Designing effective visuals requires careful consideration to ensure accessibility for all learners. Drawing from my own experience with minor color vision deficiency, I recognize common pitfalls, such as relying solely on color contrast to convey information. With color blindness affecting 8\% of males and 0.5\% of females~\cite{TypesColourBlindness}, at least one individual in a typical classroom is likely to encounter challenges in interpreting such visuals. A lecture I delivered on “Color Blindness\textendash{}Aware Scientific Visualization” highlighted simple, impactful solutions to improve accessibility, which many students found eye-opening. This experience underscored the broader role of instructors in fostering inclusivity. I am committed to creating a classroom environment that is physically, cognitively, and culturally accessible, using inclusive language and diverse examples while regularly reflecting on and refining my teaching practices.

During my senior \myDegree{} and postdoc time, I have also taken on significant leadership roles for many of our major research projects—these roles often required diffuse mentorship across large collaborative efforts in addition to highly focused, fast-paced teaching aimed at quickly bringing new students up to speed. At \mySchoolShort{}, I have mentored over ten graduate and undergraduate students, guiding them to significant milestones such as publishing their first papers and securing industry positions. These experiences have equipped me with essential skills to train successful students as a teacher and a principal investigator.

\section*{Teaching Plan}

At \appSchoolShort{}, I plan to leverage the lessons learned through trial and error from my previous years of teaching experience and pragmatically test new ideas from statistically-backed contemporary literature~\cite{deslauriersMeasuringActualLearning2019,bathgatePerceivedSupportsEvidencebased2019} in order to provide an environment which fosters maximal growth across all students, independent of background and learning style. I am both prepared and qualified to teach undergraduate/graduate courses in focused areas such as \emph{integrated photonics}, \emph{optical interconnects}, and \emph{electronic-photonic design automation}, as well as core courses in electrical and computer engineering such as \emph{semiconductor devices}, and \emph{digital logic design}. Believing that the preparation for teaching materials deepens my own understanding of the subject matter, I am also open to teaching courses beyond my immediate expertise, such as \emph{computer architecture}. I eagerly anticipate the opportunity to extend and refine my teaching and mentoring practices at \appSchoolDeptShort{}.
