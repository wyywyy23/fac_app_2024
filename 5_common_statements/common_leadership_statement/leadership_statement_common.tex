Throughout my academic and professional journey, I have pursued leadership roles that advance research innovation, foster collaboration, and empower diverse individuals to achieve their fullest potential. My leadership philosophy is built on three key pillars: driving cutting-edge advancements, nurturing next-generation researchers, and cultivating an inclusive environment.

\section*{Leadership in Research}

My research leadership has focused on redefining chip-to-chip connectivity and computing paradigms through integrated photonics. As the lead photonics designer in the DARPA PIPES program, I spearheaded the development of a hybrid 2.5D/3D photonic I/O chiplet, achieving a two-order-of-magnitude enhancement in bandwidth density and energy efficiency. These achievements laid the foundation for several successful grants, including those funded by SRC JUMP 2.0 and Samsung, enabling extensive collaborations with leading industrial partners and institutions such as Intel, MIT, and Cornell.

Beyond specific projects, I have been instrumental in advancing the silicon photonics ecosystem, co-leading large-scale design and tape-out efforts. My contributions to the development of open process design kits (PDKs) and collaborations with U.S. and international foundries have helped bridge the gap between academic research and commercial scalability. These endeavors demonstrate my ability to integrate technical vision with strategic leadership, driving impactful innovation at the intersection of photonics, electronics, and systems engineering.

\section*{Leadership in Teaching and Mentorship}

Teaching and mentorship are central to my leadership philosophy, as I believe they are pivotal to cultivating the next generation of scientists and engineers. My experience spans roles as a teaching assistant for both lecture-based and lab-intensive courses. In the latter, I guided students through hands-on debugging and design challenges. My efforts not only ensured successful project demonstrations but also inspired long-term professional growth among my students.

As a mentor, I have guided over ten graduate and undergraduate students, helping them achieve key milestones such as publishing their first lead-author papers and securing competitive industry positions. My mentorship approach emphasizes intellectual growth, critical thinking, and personal development. For example, during Columbia's Summer Undergraduate Research Experience (SURE) program, I worked closely with a student from a non-traditional academic path, tailoring my mentorship to their unique needs. This enabled them to independently present their research findings by the end of the program\textemdash{}a testament to my commitment to individualized support and student empowerment.

\section*{Leadership in Diversity and Inclusion}

Diversity and inclusion are integral to my vision of leadership. Recognizing the barriers faced by underrepresented groups in STEM, I actively engage in initiatives that broaden participation and create equitable opportunities. At the CUbiC Center under SRC JUMP 2.0, I contributed to increasing the representation of historically marginalized groups, designing accessible research projects and inclusive mentoring programs. These efforts were recognized as a model for other JUMP Centers to emulate, underscoring my ability to integrate diversity goals with organizational strategy.

My understanding of diversity extends beyond traditional identity indicators to include educational and life experiences, such as those of first-generation students, transfer students, and individuals re-entering education. This perspective informs my commitment to fostering environments where students from all backgrounds feel valued and supported. For example, I have delivered lectures on inclusive scientific visualization, raising awareness of accessibility challenges such as color vision deficiencies and proposing actionable solutions.

\section*{Vision for Future Leadership}

As I establish my research group at \appSchoolDeptShort{}, my vision is to lead transformative advancements in photonics-empowered computing while building a vibrant and inclusive academic community. My future research aims to integrate photonics into heterogeneous systems, leveraging its analog and digital capabilities to perform computation during data movement. This paradigm shift addresses fundamental challenges in AI/machine learning workloads and hyperscale computing, where efficient communication is as critical as computation.

To realize this vision, I plan to expand collaborations across academia, industry, and funding agencies, leveraging my network to bridge technical expertise with real-world applications. I will also prioritize the development of interdisciplinary tools and methodologies, equipping the next generation of researchers to tackle complex problems at the convergence of photonics, electronics, and computing systems. Furthermore, I aim to implement inclusive practices in my group by establishing mentorship frameworks that address the diverse needs of students and early-career researchers.

In summary, my leadership journey is defined by a commitment to innovation, mentorship, and inclusivity. By fostering a culture of excellence and equity, I aim to contribute to \appSchool{}'s leading role in advancing knowledge and empowering the next generation of leaders in STEM.