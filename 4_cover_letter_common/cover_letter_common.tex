I am pleased to offer my application for the \appPosition{} position \appJobID{} in the \appDept{} at the \appSchool{}. I am currently a \myTitle{} in the \myDept{} at the \mySchoolShort{}, mentored by \mySuper{}. I obtained my \myDegree{} degree in \myMajor{} from the \myPhDSchool{} in \myPhDYear{}, co-advised by \myAdvisor{} and \myCoAdvisor{}. I believe my interdisciplinary research background in silicon photonics (SiPh) spanning \appArea{} makes me uniquely suited for this position to provide a natural, complementary fit to the existing faculty at \appSchoolDeptShort{}.

My research interests broadly lie in \textbf{large-scale integrated photonic systems}, with a particular focus on transforming future computing paradigms. With the rise of large-scale artificial intelligence (AI) and machine learning (ML), data centers and high-performance computing systems have lagged in providing enough communication bandwidth between nodes to match the exponentially growing computational demands. My research at \mySchoolShort{} aimed at alleviating this bottleneck, and enabling continued system scaling, by equipping compute chips with \textbf{embedded SiPh optical input/output (I/O)}, achieving manifold improvements in bandwidth density and energy efficiency. Notably, this research led to the first demonstration of a hybrid 2.5D/3D integration of photonic I/O chiplets with flip-chip bonded electronic drivers and a state-of-the-art compute chip, packing 96\,Tbps bidirectional bandwidth in a single package with a 4\,Tbps/mm shoreline bandwidth density. The enormous co-design effort involved in this research was recognized by publications in prestigious venues spanning photonics, electronics, and packaging domains, including \emph{OFC}, \emph{CICC}, and \emph{IEEE T-CPMT}, and the successful transition into Phase 3 of the \emph{DARPA PIPES} program. Promising of unprecedented interconnect scalability and flexibility enabled by massive wavelength parallelism, this research also sparked \textbf{network-application co-optimization}\textemdash{}a direction I have pursued since my \myDegree{}\textemdash{}to accelerate AI/ML workloads at the system level with only a fraction of energy consumption. This framework was pivotal in the successful application for a \$35M \emph{SRC JUMP 2.0} grant to develop revolutionary system connectivity.

Moving forward, I envision even greater interconnect density and adaptability becoming mandatory in computing systems facing an increasingly dynamic and heterogeneous data landscape. By routing lightwaves in a third dimension out-of-plane, a dense \textbf{3D optical I/O} promises to shift the paradigm of chip-to-chip connectivity, surpassing the limits of conventional planar photonics. The conceptual design of such a system, which I helped formulate for a showcase at the 2023 DARPA ERI Summit, was well received and followed up with a compelling proposal submitted to the \emph{DARPA HAPPI} program. This paradigm shift in connectivity will also enable new classes of interconnect functionalities, such as \textbf{photonics-enabled computing and signal processing}, for a variety of applications beyond data communication. I have led or participated in the writing of several successfully funded proposals in these directions from both government and industry sources, and have maintained close collaborations with leading researchers from multiple institutions and companies on various ongoing projects. I believe that these experiences have prepared me well to establish a strong externally-funded research group at \appSchoolDeptShort{} with highly collaborative network spanning academia, industry, and government. I additionally envision the potential for exciting interdisciplinary collaborations with the current faculty at \appSchoolShort{}, including:
\begin{enumerate*}[label=(\roman*)]
    \appCollab{}
\end{enumerate*}

I am truly enthusiastic about the opportunity to work alongside the world-class faculty and students at \appSchoolDeptShort{} with the shared goal of tackling the grand challenges in future computing systems. I appreciate your consideration and look forward to hearing back from you.